\documentclass{beamer}
\usetheme{Berkeley}
 
\usepackage[utf8]{inputenc}
\usepackage{amssymb}
\usepackage{amsmath} %piecewise function%
 
 
%Information to be included in the title page:
\title{Modular Forms}
\author{Robert Sweeney Blanco}
\institute{Advisor: Brandon Williams}
\date{May 9, 2018}
 
 
 
\begin{document}
 
\frame{\titlepage}
 
 %Defining Modular Forms%
\begin{frame}
\frametitle{Defining Modular Forms}
In this presentation, I will be working with the modular group $SL_2(\mathbb{Z}) = \Gamma_1$
\begin{block}{Definition: Modular form}
$f(z)$ is a modular form of weight $k$ if 
\begin{enumerate}
\item  f is holomorphic on $\mathbb{H}$\\
\item f continues to be holomorphic as $\Im(z) \longrightarrow \infty$
\item $f(\frac{az+b}{cz+d}) = (cz+d)^k f(z) \  \ \ \forall \begin{bmatrix}
a & b \\
c & d \\
\end{bmatrix} 
\in \Gamma_1$
\end{enumerate}
\end{block}
\end{frame}

%Simplifying the full Modular Group%
\begin{frame}
\frametitle{Simplifying the full Modular Group}
\begin{block}{Theorem}
$\Gamma_1$ is a group generated by $S = \begin{bmatrix}
0 & 1 \\
-1 & 0 \\
\end{bmatrix}$
and $T = \begin{bmatrix}
1 & 1 \\
0 & 1 \\
\end{bmatrix}$
\end{block}
This simplifies proving a function is a modular form when working with the full modular group because now we only need to check that the desired behavior for composition with Mobius transformations from the matrix group works for the transformations corresponding with $S$ and $T$. That is $f(S \cdot z))=f(\frac{-1}{z}) = z^kf(z)$ and $f(T \cdot z)=f(z+1)=f(z)$. Note this means modular forms are periodic and have Fourier series!
\end{frame}

%Dimension Formula%
\begin{frame}

\frametitle{Dimension of $M_k(\Gamma_1)$}
Modular forms of a fixed weight form a vector space over the Complex numbers. The dimension of $M_k(\Gamma_1)$ is $0$ for all negative and odd values of $k$. Otherwise the dimension is given by the following formula:
\[   
dim(M_k(\Gamma_1)) = 
     \begin{cases}
       \lfloor \frac{k}{12} \rfloor + 1 & \quad \text{if} \  k \not \equiv  2 \ mod(12) \\
       \lfloor \frac{k}{12} \rfloor & \quad \text{if} \  k \equiv  2 \ mod(12) \\
     \end{cases}
\]
\end{frame}

%Eisenstein Series%
\begin{frame}

\frametitle{Defining Eisenstein Series}
There are different ways of defining the Eisenstein Series, but they all differ by a constant. Here is a common definition for series of even weight $k>2$: 
\begin{block}{Eisenstein Series}
$$G_k(z) = \sum_{\substack{(m,n) \in \mathbb{Z}^2 \\ (m,n) \neq (0,0)}}  \frac{1}{(m z + n)^k}$$
\end{block}
This series converges absolutely for $k>2$.

\end{frame}

%Proving Eisenstein Series transforms under T%
\begin{frame}
\frametitle{Transformation under $T$}

\begin{align*}
& G_k(z+ 1) = \sum_{\substack{(m,n) \in \mathbb{Z}^2 \\ (m,n) \neq (0,0)}}  \frac{1}{(m (z+1) + n)^k} \\
& = \sum_{\substack{(m,n) \in \mathbb{Z}^2 \\ (m,n) \neq (0,0)}}  \frac{1}{(m z+ m +  n)^k} \\
& \text{Define } \mu=m+n \ (\text{Note } (m,n) \neq (0,0) \iff (m, \mu) \neq (0,0)) \\
& =  \sum_{\substack{(m,\mu) \in \mathbb{Z}^2 \\ (m,\mu) \neq (0,0)}}  \frac{1}{(m z+ \mu)^k} = G_k(z)
\end{align*}


\end{frame}

%Proving Eisenstein Series transforms under S%
\begin{frame}
\frametitle{Transformation under $S$}

\begin{align*}
&G_k \left( \frac{-1}{z} \right) = \sum_{\substack{(m,n) \in \mathbb{Z}^2 \\ (m,n) \neq (0,0)}}  \frac{1}{(m (\frac{-1}{z}) + n)^k} \\
&= \sum_{\substack{(m,n) \in \mathbb{Z}^2 \\ (m,n) \neq (0,0)}}  \frac{z^k}{(-m + n z)^k} = z^k \sum_{\substack{(m,n) \in \mathbb{Z}^2 \\ (m,n) \neq (0,0)}}  \frac{1}{(-m + n z)^k} \\
& \text{Define} (\mu, \eta) = (n, -m) (\text{Note} (m,n) \neq (0,0) \iff (\mu, \eta) \neq (0,0)) \\
& = z^k \sum_{\substack{(\mu, \eta) \in \mathbb{Z}^2 \\ (\mu, \eta) \neq (0,0)}}  \frac{1}{(\mu z + \eta)^k} = z^k G_k(z)
\end{align*}

\end{frame}

%Proving Eisenstein Series satisfy growth condition%
\begin{frame}
\frametitle{Proving the Growth Condition}
\begin{align*}
& \lim_{\Im(z) \to \infty} \sum_{\substack{(m,n) \in \mathbb{Z}^2 \\ (m,n) \neq (0,0)}}  \frac{1}{(m z + n)^k} \\
& = \lim_{\Im(z) \to \infty} \left[ \sum_{\substack{n \in \mathbb{Z} \\ n \neq 0}} \frac{1}{n^k} + \sum_{\substack{(m,n) \in \mathbb{Z}^2 \\ m \neq 0}}  \frac{1}{(m z + n)^k} \right] \\
& = \sum_{\substack{n \in \mathbb{Z} \\ n \neq 0}} \frac{1}{n^k} = 2 \sum_{\substack{n \in \mathbb{N} \\ n \neq 0}} \frac{1}{n^k}  = 2\zeta(k)
\end{align*}
\end{frame}

%Proving Eisenstein Series are modular forms%
\begin{frame}
\frametitle{Proof Conclusion}
\begin{enumerate}
\item Since the series converges uniformly on any compact subset of $\mathbb{H}$, it is holomorphic
\item The series transforms properly for $S$ and $T$ $\Rightarrow$ satisfies Modularity Condition
\item The series satisfies the growth condition
\end{enumerate}


$\therefore$ $G_k$ is a modular form for $k > 2$. Moreover, it is a modular form of weight $k$.

\end{frame}

%Fourier Expansion of Eisenstein Series%
\begin{frame}
\frametitle{Fourier Expansion for Eisenstein Series}
Since Eisenstein Series for even $k>2$ are modular forms, they have a unique Fourier expansion. 
\begin{block}{Fourier Definition of Eisenstein Series}
$$E_k(z) = 1 - \frac{2k}{B_k} \sum_{d,n \geq 1} n^{k-1} e^{2\pi i dnz} $$\\
Where $B_k$ is the $k$th Bernoulli number.
\end{block}
\end{frame}

%Eisenstein Identities%
\begin{frame}
\frametitle{Eisenstein Identities}
The set of modular forms $M_{*}(\Gamma_1)$ form a ring. Moreover, the weight of the product of  modular forms is the sum of their weights. This paves the way for several easy identities: \\
\begin{enumerate}
\item $E_4^2$ is a modular form of weight $8$, both $E_4^2$ and $E_8$ have Fourier series that start with 1, and $dim(M_8(\Gamma_1)) = 1 \Rightarrow E_4^2 \equiv E_8$  
\item  $E_4 \cdot E_6$ is a modular form of weight 10, the product of their Fourier series has a leading term of 1, and $dim(M_{10}(\Gamma_1)) = 1 \Rightarrow E_4 \cdot E_6 \equiv E_{10}$
\end{enumerate}
\end{frame}

%Definition of the Discriminant function%
\begin{frame}
\frametitle{Discriminant Function}
A famous modular form of weight 12 is the Discriminant Function. 
\begin{block}{Discriminant Function}
$$\Delta(z) = e^{2 \pi i z} \prod_{n=1}^{\infty} (1-e^{2 \pi i n z})^{24}$$
\end{block}
\end{frame}

%Fourier Series of Discriminant function%
\begin{frame}
\frametitle{Fourier Expansion of Discriminant Function}
Since the Discriminant function is a modular form, it has a Fourier expansion. Ramanujan tau function $\tau(n)$ is defined as the coefficients of the series. \\
$$\Delta(z) = \sum_{n = 1}^{\infty} \tau(n) e^{2n\pi i z} = e^{2 \pi i z} - 24 e^{4\pi i z} + 252 e^{6 \pi i z} -...$$
\end{frame}

%Discriminant Function identity%
\begin{frame}
\frametitle{Discriminant Function identity}
Note that $E_4^3 = (1 + 240e^{2\pi i z} + ...)^3 = 1+720e^{2\pi i z}+...$ and $E_6^2 = (1-504e^{2\pi i z}+...)^2 = 1 - 1008 e^{2n\pi i z}+...$ are not scalar multiples of each other. Since $dim(M_{12}(\Gamma_1)) = 2$, both $E_4^3$ and $E_6^2$ span $M_{12}(\Gamma_1)$, and thus there is a linear combination of them that equal $\Delta(z)$. Since the first term in the Fourier series of $\Delta(z)$ is $e^{2\pi i z}$, it must be some scalar multiple of the difference of  $E_4^3$ and $E_6^2$ (to get rid of the leading one). The first term of $E_4^3 - E_6^2$ is $1728e^{2\pi i z}$, so we need to scale down by $1728$, which will match the first term of $\Delta(z)$ and thus we are done. $\therefore \Delta(z) = \frac{E_4^3 - E_6^2}{1728}$. 
\end{frame}

\end{document}
