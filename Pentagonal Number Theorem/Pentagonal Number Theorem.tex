\documentclass{beamer}
\usetheme{Berkeley}
 
\usepackage[utf8]{inputenc}
\usepackage{amssymb}
\usepackage{amsmath} %piecewise function%
 
 
%Information to be included in the title page:
\title{Modular Forms}
\author{Robert Sweeney Blanco}
\institute{Advisor: Brandon Williams}
\date{December 5, 2017}
 
 
 
\begin{document}
 
\frame{\titlepage}
 
\begin{frame}
\frametitle{Defining Modular Forms}
In this presentation, I will be working with the modular group $SL_2(\mathbb{Z}) = \Gamma_1$
\begin{block}{Definition: Modular form}
$f$ is a modular form of weight $k$ if 
\begin{enumerate}
\item  f is holomorphic on $\mathbb{H}$\\
\item f continues to be holomorphic as $z \longrightarrow i\infty$
\item $f(\frac{az+b}{cz+d}) = (cz+d)^k f(z) \  \ \ \forall \begin{bmatrix}
a & b \\
c & d \\
\end{bmatrix} \in \Gamma_1$
\end{enumerate}
\end{block}
\end{frame}

\begin{frame}
\frametitle{Dimension of $M_k(\Gamma_1)$}
Modular forms of a fixed weight form a vector space over the Complex numbers. The dimension of $M_k(\Gamma_1)$ is $0$ for all negative and odd values of $k$. Otherwise the dimension is given by the following formula:
\[   
dim(M_k(\Gamma_1)) = 
     \begin{cases}
       \lfloor \frac{k}{12} \rfloor + 1 & \quad \text{if} \  k \not \equiv  12 \ mod(12) \\
       \lfloor \frac{k}{12} \rfloor & \quad \text{if} \  k \equiv  12 \ mod(12) \\
     \end{cases}
\]
\end{frame}

\end{document}